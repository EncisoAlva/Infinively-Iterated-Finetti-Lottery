\documentclass{article}
\usepackage{graphicx} % Required for inserting images

%%%%%%%%%%%%%%%%%%%%%%%%%%%%%%%%%%%%%%%%%%%%%%%%%%%%%%%%%%%%%%%%%%%%%%%%%%%%%%%
%  PACKAGES
%%%%%%%%%%%%%%%%%%%%%%%%%%%%%%%%%%%%%%%%%%%%%%%%%%%%%%%%%%%%%%%%%%%%%%%%%%%%%%%

\usepackage{amsmath}
\usepackage{amsfonts}

\usepackage{appendix}

%%%%%%%%%%%%%%%%%%%%%%%%%%%%%%%%%%%%%%%%%%%%%%%%%%%%%%%%%%%%%%%%%%%%%%%%%%%%%%%
%  METADATA
%%%%%%%%%%%%%%%%%%%%%%%%%%%%%%%%%%%%%%%%%%%%%%%%%%%%%%%%%%%%%%%%%%%%%%%%%%%%%%%

\title{Expected Probabilities of Infinitely Iterated de Finetti Lotteries via Matrix Decomposition}
\author{Julio Cesar Enciso-Alva}
\date{February 2024}

%%%%%%%%%%%%%%%%%%%%%%%%%%%%%%%%%%%%%%%%%%%%%%%%%%%%%%%%%%%%%%%%%%%%%%%%%%%%%%%
%  MACROS
%%%%%%%%%%%%%%%%%%%%%%%%%%%%%%%%%%%%%%%%%%%%%%%%%%%%%%%%%%%%%%%%%%%%%%%%%%%%%%%

\newtheorem{definition}{Definition}

\newcommand{\N}{\mathbb{N}}
\newcommand{\R}{\mathbb{R}}
\newcommand{\aabs}[1]{\left| #1 \right|}
\newcommand{\card}[1]{\left| #1 \right|}
\newcommand{\sset}[1]{\left\{ #1 \right\}}
\newcommand{\ppar}[1]{\left( #1 \right)}
\newcommand{\spar}[1]{\left[ #1 \right]}

%%%%%%%%%%%%%%%%%%%%%%%%%%%%%%%%%%%%%%%%%%%%%%%%%%%%%%%%%%%%%%%%%%%%%%%%%%%%%%%
%%%%%%%%%%%%%%%%%%%%%%%%%%%%%%%%%%%%%%%%%%%%%%%%%%%%%%%%%%%%%%%%%%%%%%%%%%%%%%%

\begin{document}

\maketitle

\begin{abstract}
    TODO
\end{abstract}

%%%%%%%%%%%%%%%%%%%%%%%%%%%%%%%%%%%%%%%%%%%%%%%%%%%%%%%%%%%%%%%%%%%%%%%%%%%%%%%
%  INTRODUCTION
%%%%%%%%%%%%%%%%%%%%%%%%%%%%%%%%%%%%%%%%%%%%%%%%%%%%%%%%%%%%%%%%%%%%%%%%%%%%%%%

\section{Introduction}

A lottery can be defined informally as randomly selecting a ticket from a set of tickets. 
%
Real-world lotteries work with finite sets of tickets and are typically associated with randomized decisions --including monetary prizes. 
%
These finite lotteries can be used to illustrate probability concepts such as equiprobability or sampling with and without substitution.

Wenmackers and Horsten \cite{fair_infinite_lottery} describe with great detail the generalization of fair lotteries with infinite tickets.
%
They propose that a fair lottery should keep the following properties:
\begin{enumerate}
    \item Each single ticket shouldn't have a higher probability of being selected than any other ticket.
    \item Any individual ticket should be able to be selected.
    \item The probability of selecting a group of tickets should be equal to the sum of the probability of choosing each individual ticket.
    \item The labeling of the tickets is independent of the outcome.
\end{enumerate}

The case of a fair lottery over the set of natural numbers is known as the de Finneti Lottery in honor of Bruno de Finetti.
%
A description of the numerous implications of modeling the de Finetti Lottery within the framework of standard Probability Theory is beyond the scope of this work, and the interested reader should refer to the paper of Wenmackers and Horsten \cite{fair_infinite_lottery}.


Hess and Polisety \cite{hess2023} proposed a variation of the de Finetti Lottery in which the tickets corresponding to odd numbers are selected with replacement, and the tickets from even numbers are selected without replacement;
this lottery is then iterated once per each ticket present at the beginning.
%
For ease of notation, this process will be referred to as the Infinitely-Iterated de Finetti Lottery (IIFL) during this text.

After defining the IIFL, Hess and Polisety \cite{hess2023} investigated the ratios of tickets related to odd and even numbers after the iterations were performed.
%
Later, they obtain these ratios after `half' of the iterations.

In this work, I propose the concept of Finitely-Iterated de Finetti Lottery (FIFL) as a framework to study quantities related to IIFL.
%
By using FIFL, I was able to replicate and generalize the findings of Hess and Polisety \cite{hess2023}.

%%%%%%%%%%%%%%%%%%%%%%%%%%%%%%%%%%%%%%%%%%%%%%%%%%%%%%%%%%%%%%%%%%%%%%%%%%%%%%%
%  DEFINITIONS
%%%%%%%%%%%%%%%%%%%%%%%%%%%%%%%%%%%%%%%%%%%%%%%%%%%%%%%%%%%%%%%%%%%%%%%%%%%%%%%

\section{Formal definitions}

I will keep the informal notion of a fair lottery as the random selection of a ticket from a set of tickets in compliance with the fairness properties described in Section 1;
a proper formalization of this concept is beyond this text, and thus the reader should refer to the paper by Wenmackers and Horsten \cite{fair_infinite_lottery}.

In the paper by Hess and Polisety \cite{hess2023}, the lottery is iterated a countable infinite number of times; tickets corresponding to even numbers are selected without replacement, and tickets from odd numbers are selected with replacement.
%
This process is iterated one time per each natural number.

Hess and Polisety investigated the ratio (even tickets)/(odd tickets) tickets after all the iterations were performed;
%and found it to be 
%$W\ppar{e^{-1}}/\ppar{1+W\ppar{e^{-1}}}$, where $W$ is the Lambert W function.
%
the concept of the ratio between infinite sets was not explored in that paper but rather considered to be an intuitive generalization from finite sets.
%
A compatible definition is provided here.

\begin{definition}[Natural density for a subset of $\N$]
Let $A \subseteq \N$ be a countable set.
%and define the following sequence
%\begin{equation}
%    m_N\ppar{A} = \frac{1}{N} \card{ \sset{a\in A; a\leq N} }.
%\end{equation}
If the following limit is well-defined, we call it the density of $A$,
\begin{equation}
    m\ppar{A} = 
    %\lim_{N\leftarrow \infty} m_N\ppar{A}.
    \lim_{N\rightarrow \infty} \frac{1}{N} \card{ \sset{a\in A; a\leq N} }.
\end{equation}
\end{definition}

In the context of lotteries, this definition fails property 4 established by Wenmackers and Horsten since it doesn't guarantee that the density of any set will be kept after a permutation of labels.
%
This definition is used for compatibility with the work of Hess and Polisety.

Similarly, the proposed definition for the iterative de Finetti Lottery is compatible on an operative level; the general initial conditions are already considered.

\begin{definition}[Finitely-Iterated Lottery]
Let $A \subseteq \N$ be a countable set with $m\ppar{A} = p \in [0,1]$, and let $\alpha\geq 0$ and $N\in \N$ be parameters. 

Construct the sets $A_0 = \sset{a\in A; a\leq N}$, $B = \sset{1, 2, \dots, N} - A_0$.
%
The iterative lottery consists of the following steps, iterated over $n$ with $1\leq n\leq \alpha N$:
\begin{enumerate}
    \item Select randomly $x \in A_n \cup B$, with all elements being equiprobable.
    \item Construct $A_{n+1}$ as follows
    \begin{equation}
        A_{n+1} = \begin{cases}
            A_n-\sset{x}, &\text{if } x\in A_n, \\
            A_n, &\text{otherwise}.
        \end{cases}
    \end{equation}
\end{enumerate}
\end{definition}

With this definition at hand, the idea of iterating the lottery `as many times as natural numbers' is equivalent to observing the behavior of $A_{N}$ as $N \rightarrow \infty$ with $p=\frac{1}{2}$ and $\alpha=1$.
%
The parameter $\alpha$ is tied to the follow-up questions related to performing only half of the iterations.
%, which in this framework only requires $\alpha = \frac{1}{2}$.


Since our interest in the lotteries depends on the natural density of the outcomes, instead of the actual members of the set.
%
It is then convenient to redefine the finite de Finetti lottery in terms of cardinalities.

\begin{definition}[Finitely-Iterated de Finetti Lottery (FIFL)]
Let $N\in \N$, $\pi \in [0,1]$, and $\beta \in [0, \infty)$ parameters.

The iterative lottery's result is defined by the following sequence
\begin{align}
    S\ppar{0; N, \pi} &= \max\sset{k\in \N; k \leq \pi N}, \\
    S\ppar{n+1; N, \pi} &= S\ppar{n; N, \pi} - \text{Bernoulli}\ppar{\frac{S\ppar{n; N, \pi}}{N+S\ppar{n; N, \pi}}},
\end{align}
with $1\leq n\leq \beta N$; Bernoulli$\ppar{p}$ is a random variable that is 0 with probability $p$, and it is 1 with probability $1-p$.
\end{definition}

The parameters $\pi = \frac{p}{1-p}$ and $\beta = \ppar{1+\pi} \beta$ are introduced for ease of computation and in order to make sure that
\begin{align}
    p &=
    \frac{\card{A_0}}{\card{A_0 \cup B}} = \frac{S_0}{S_0+N},
    \\
    \alpha &=
    \frac{\text{iterations}}{\card{A_0 \cup B}} =
    \frac{\text{iterations}}{S_0+N}.
\end{align}

Needless to say that $S$ is a random process.
%
We are further interested in the expected value of $S$, or more specifically the expected density.
\begin{equation}
    m_N\ppar{\pi, \beta} = E\ppar{\frac{1}{N}S\ppar{\beta N; N, \pi }}
\end{equation}

The convergence of $S$ as $N \rightarrow \infty$ is not explored in this paper, but instead only the behavior of $m_N$. 
%
In particular, we consider the following quantity whenever it is well-defined
\begin{equation}
    \mu\ppar{\pi, \beta} = \lim_{N\infty} m_N\ppar{\pi, \beta}
\end{equation}

For readability, a second version of $\mu$ is considered using the original variables $p$ and $\alpha$. 
%
Recall that $p$ is the natural density of the set to be selected on the lottery, while $\alpha$ is the ratio of iterations/density of $\N$.
%
This density function with the original variables would be
\begin{equation}
    \mu^*\ppar{p, \alpha} = \mu\ppar{\frac{p}{1-p}, \frac{1}{1-p} \alpha}
\end{equation}

Within this framework,
the findings from Hess and Polisety \cite{hess2023} can be described as the following
\begin{align}
\mu^*\ppar{\frac{1}{2}, 1}
    %\lim_{N \rightarrow \infty} E\ppar{\frac{1}{N}S\ppar{2 N; N, \frac{1}{2}}} 
    &=
    \frac{W\ppar{e^{-1}}}{1+W\ppar{e^{-1}}} \approx 0.2178
    \\
\mu\ppar{\frac{1}{2}, \frac{1}{2}}
    %\lim_{N \rightarrow \infty} E\ppar{\frac{1}{N}S\ppar{ N; N, \pi}} 
    &=
    \frac{W\ppar{1}}{1+W\ppar{1}} \approx 0.3619
\end{align}

%In the following sections, we generalize this result by proposing the following
%\begin{align}
%    \mu\ppar{\pi, \beta} &= \frac{W\ppar{g\ppar{\pi, \beta}}}{1+W\ppar{g\ppar{\pi, \beta}}}
%    \\
%    g\ppar{\pi, \beta} &=
%    \pi \exp{\ppar{\pi - \beta}}
%\end{align}

The proposed result is that
\begin{align}
    {\mu^*}\ppar{p, \alpha} &= \frac{W\ppar{h\ppar{p, \alpha}}}{1+W\ppar{h\ppar{p, \alpha}}}
    \\
    h\ppar{p, \alpha} &=
    \frac{p}{1-p} \exp{\ppar{\frac{p-\alpha}{1-p}}}
\end{align}
subject to the following
\begin{align}
    \alpha > 1 + \ppar{1-p} \ln{\ppar{\frac{p}{1-p}}} 
\end{align}

\section{Model as a Markov Process}

As described in the previous section, $S(\bullet; N, \pi)$ is clearly a Markov Process.
%
The states of this process can be mapped to the possible values of $m_N$, the quantity of interest, as
\begin{equation}
    \frac{1}{N} S(t; N, \pi) \in \sset{\frac{k}{N+k}; k = 0, 1, 2, \cdots}
\end{equation}
for $t\in \N$. 
%
Since there are countably infinite states for this process, we can define a state vector $Z_N \in \R^{\N\times 1}$ as
\begin{equation}
    \spar{Z_N(t)}(k) = Pr\ppar{\frac{1}{N} S(t; N, \pi) = \frac{k}{N+k}}, \text{ for } k=0,,1, 2, \dots
\end{equation}
with the initial condition
\begin{equation}
    \spar{Z_N(0)}(k) = \spar{e_{\pi N}} = \begin{cases}
        1, &\text{if } \max\sset{k\in \N; k \leq \pi N} \\
        0, &\text{otherwise.}
    \end{cases}
\end{equation}
with $e_\tau$ the $\tau$-th canonical vector.

The evolution of this system is given by the following condition
\begin{equation}
    Pr\ppar{\frac{1}{N} S(t+1; N, \pi) = x \;\middle|\; \frac{1}{N} S(t; N, \pi) = \frac{k}{N+k}} = 
    \begin{cases}
        \frac{N}{N+k}, &\text{for } x= \frac{k}{N+k} \\
        \frac{k}{N+k}, &\text{for } x= \frac{k-1}{N+k-1} \\
        0, &\text{otherwise.}
    \end{cases}
    \label{eq:evolution}
\end{equation}
which can be encoded using a multiplication of the state vector, $Z_N$, with a transition matrix, $M_N\in \R^{\N\times \N}$, defined as
\begin{equation}
    \spar{M_N}(j,k) = 
    \begin{cases}
        \frac{N}{N+k}, &\text{if } j= k, k>0 \\
        \frac{k}{N+k}, &\text{if } j= k-1, k>0 \\
        1, &\text{if } j= k=0 \\
        0, &\text{otherwise.}
    \end{cases}
\end{equation}
And thus, the evolution equation \eqref{eq:evolution} can be rewritten as
\begin{equation}
    Z_N(t+1) = M_N Z_N(t) = \spar{M_N}^t Z_N(0).
\end{equation}

The original goal, studying the behavior of $m_N$, is connected by writing
\begin{align}
    m_N(\pi, \beta) 
    &= 
    E\ppar{\frac{1}{N}S\ppar{\beta N; N, \pi }}
    =
    \sum_{k=0}^{\infty} \frac{k}{N+k} \spar{Z_N(\beta N)}(k)
    =
    \nu^T Z_N(\beta N)
    \label{eq:weighted}
\end{align}
where $\nu_N \in \R^{\N \times 1}$ is a vector given by
\begin{equation}
    \spar{\nu_N}(k) = \frac{k}{N+k}, \text{ for } k=0,1, 2, \dots
\end{equation}

Notice that, in every finite case, the infinite sum in \eqref{eq:weighted} can be truncated at $k\leq \beta N$ since $Z_N$ is zero for those values.
%
However, it is convenient to keep it like that since it allows us to consider all possible values of $\pi$ by changing only the initial conditions.
%
This becomes clear by rewriting $m_N$ as
\begin{equation}
    m_N(\pi, \beta) = \nu_N^T \spar{M_N}^{\beta N}\, e_{\pi N}.
    \label{eq:mat_mult_base}
\end{equation}

%%%%%%%%%%%%%%%%%%%%%%%%%%%%%%%%%%%%%%%%%%%%%%%%%%%%%%%%%%%%%%%%%%%%%%%%%%%%%%%
%  EXPECTED VALUE
%%%%%%%%%%%%%%%%%%%%%%%%%%%%%%%%%%%%%%%%%%%%%%%%%%%%%%%%%%%%%%%%%%%%%%%%%%%%%%%

\section{Results}

Our strategy, whose details are carried out in detail in the appendix, is to compute a sort of eigendecomposition of $M_N$ and use it to compute $m_N$ efficiently.
%
The existence of such objects is guaranteed since $M_N$ is lower diagonal, and thus its eigenvalues are trivially found as
\begin{equation}
    \lambda^{(N)}_k = \frac{N}{N+k}, \text{ for } k=0,1, 2, \dots
\end{equation}

The $k$-th eigenvalue of $M_N$, which we now refer as $V_k^{(k)}$, is defined by the following property
\begin{equation}
    M_N V_k^{(N)} = \lambda^{(N)}_k V_k^{(N)}
\end{equation}

As it is shown in the appendix \ref{ap:eigenvalues}, those eigenvalues are given by
\begin{equation}
    \spar{V_k^{(N)}}(n) =
    (-1)^{k-n} \ppar{\frac{N+k}{N}}^{k-n} \cdot \frac{N+n}{N+k} \binom{k}{n}
\end{equation}

With this result at hand, the eigendecomposition of $M_N$ is given by
\begin{align}
    M_N &= \spar{V_N}^{-1} \Lambda_N V_N 
    \label{eq:eigendecomp}
    \\
    V_N &= \begin{bmatrix}
        V_0^{(N)}, & V_1^{(N)}, & \dots
    \end{bmatrix} \\
    \Lambda_N &= \text{diag}\ppar{\lambda^{(N)}_0, \lambda^{(N)}_1, \dots}
\end{align}

As it is proven in the appendix \ref{ap:inverse}, the inverse of $V_N$ exists and it is given by
\begin{align}
    \spar{V^{-1}}(n,k) &=
    \ppar{\frac{N+n}{N}}^{k-n} \binom{k}{n}
\end{align}

The objective of defining and computing the eigendecomposition of $M_N$ is to compute efficiently $m_N$.
%
In particular, from equations \eqref{eq:mat_mult_base} and \eqref{eq:eigendecomp} we have
\begin{equation}
    m_N(\pi, \beta) = \nu_N^T 
    \spar{V_N}^{-1} \spar{\Lambda_N}^{\beta N} V_N\,
    e_{\pi N}
\end{equation}

As it is proven in the appendix \ref{ap:product}, the result of this computation is given by
\begin{equation}
m_N(\pi, \beta) =
    \sum_{k=0}^{\pi N}
    (-1)^{k+1} 
    \binom{\pi N}{k}
    \ppar{\frac{k}{N+k}}^k
    \ppar{\frac{N+k}{N}}^{\spar{\pi-\beta}k N -1}
    \label{eq:eq2}
\end{equation}

As it is proven in the appendix \ref{ap:coeff}, the coefficients on equation \eqref{eq:eq2} converge pointwise with $N\rightarrow \infty$ to the following expression
\begin{equation}
    \mu(\alpha, \beta) =
    \sum_{k=0}^{\infty}
    (-1)^{k+1} \frac{k^k}{k!} 
    \ppar{\pi
    e^{\spar{\pi-\beta}} }^k
    \label{eq:mu_sum}
\end{equation}

As it is proven in the appendix \ref{ap:convergence},
the series on \eqref{eq:mu_sum} converges under the following condition
\begin{equation}
    \beta > \pi + 1 + \ln{\pi}
\end{equation}

A more compact expression for \eqref{eq:mu_sum} can be obtained using the following identity
\begin{equation}
    \frac{W\ppar{z}}{1+W\ppar{z}}
    =
    \sum_{k=0}^{\infty}
    (-1)^{k+1} \frac{k^k}{k!} 
    z^k
\end{equation}
resulting in the following 
\begin{align}
    \mu\ppar{\pi, \beta} &= \frac{W\ppar{g\ppar{\pi, \beta}}}{1+W\ppar{g\ppar{\pi, \beta}}}
    \\
    g\ppar{\pi, \beta} &=
    \pi \exp{\ppar{\pi - \beta}}
\end{align}

It is worth to rewrite this result in terms of $p$ and $\alpha$, the original variables.
\begin{align}
    {\mu^*}\ppar{p, \alpha} &= \frac{W\ppar{h\ppar{p, \alpha}}}{1+W\ppar{h\ppar{p, \alpha}}}
    \\
    h\ppar{p, \alpha} &=
    \frac{p}{1-p} \exp{\ppar{\frac{p-\alpha}{1-p}}}
\end{align}
subject to the following
\begin{align}
    \alpha > 1 + \ppar{1-p} \ln{\ppar{\frac{p}{1-p}}} 
\end{align}

\section{Discussion}

The formula on \eqref{eq:mu_sum} is consistent with the findings by Hess and Polisetty.

The convergence condition was proven using the standard convergence test for series, which should also determine the speed of convergence.
%
However, computing the required products reveals that the results converge indeed.
%
This may be caused by the unconditional convergence of the final product.

Because my primary field is Applied Math, I can't comment on the significance of these results. 

%%%%%%%%%%%%%%%%%%%%%%%%%%%%%%%%%%%%%%%%%%%%%%%%%%%%%%%%%%%%%%%%%%%%%%%%%%%%%%%
%  APPENDIX
%%%%%%%%%%%%%%%%%%%%%%%%%%%%%%%%%%%%%%%%%%%%%%%%%%%%%%%%%%%%%%%%%%%%%%%%%%%%%%%

\newpage

\appendix 

\section{Eigenvalues of Transition Matrix}
\label{ap:eigenvalues}

Recall that the transition matrix, $M_N\in \R^{\N\times \N}$, is defined as
\begin{equation}
    \spar{M_N}(j,k) = 
    \begin{cases}
        \frac{N}{N+k}, &\text{if } j= k, k>0 \\
        \frac{k}{N+k}, &\text{if } j= k-1, k>0 \\
        1, &\text{if } j= k=0 \\
        0, &\text{otherwise.}
    \end{cases}
\end{equation}

The eigenvalues of $M_N$ are given by
\begin{equation}
    \lambda^{(N)}_k = \frac{N}{N+k}, \text{ for } k=0,1, 2, \dots
\end{equation}

The $k$-th eigenvalue of $M_N$, which we now refer as $V_k^{(k)}$, is defined by the following property
\begin{equation}
    M_N V_k^{(N)} - \lambda^{(N)}_k V_k^{(N)} = 0
    \label{a:01}
\end{equation}

The $j$-th column of \eqref{a:01} is given by
\begin{equation}
    \ppar{\frac{N}{N+j}-\frac{N}{N+k} } V_k^{(N)}(j) + \ppar{\frac{j+1}{N+j+1}}V_k^{(N)}(j+1) = 0
    \label{a:induction}
\end{equation}

One particular case is the $k$-th column,
\begin{equation}
    \ppar{\frac{k+1}{N+k+1}}V_k^{(N)}(k+1) = 0
\end{equation}
from which it follows that $V_k^{(N)}(k+1) = 0$. Furthermore, by induction over \eqref{a:induction} this implies that $V_k^{(N)}(j) = 0$ for $j>k$.

For $j\leq k$, equation \eqref{a:induction} can be rewritten as
\begin{align}
    V_k^{(N)}(j) 
    &= -
    \frac{\ppar{\frac{j+1}{N+j+1}}}{\ppar{\frac{N}{N+j}-\frac{N}{N+k} }}
    V_k^{(N)}(j+1)
    \nonumber \\
    &=
    -
    \ppar{\frac{N+k}{N}} \ppar{\frac{N+j}{N+j-1}} \ppar{\frac{j+1}{k-j}} V_k^{(N)}(j+1)
\end{align}

Without loss of generality, we can define $\spar{V_k^{(N)}}(k) = 1$, and then the other entries are given by
\begin{equation}
    \spar{V_k^{(N)}}(j) =
    \begin{cases}
    (-1)^{k-j} \ppar{\frac{N+k}{N}}^{k-j} \cdot \frac{N+j}{N+k} \binom{k}{j},
    &\text{for } 0 \leq j\leq k\\
    0, &\text{otherwise.}
    \end{cases}
\end{equation}

Once the eigenvalues are found, it is possible to ensemble a matrix $V_N$ whose columns are the eigenvalues of $M_N$.
\begin{equation}
    V_N = \begin{bmatrix}
        V_0^{(N)}, & V_1^{(N)}, & \dots
    \end{bmatrix}
\end{equation}

\section{Inverse of Eigenvalues of Transition Matrix}
\label{ap:inverse}

The purpose of computing the eigendecomposition of $M_N$ is to use the identity
\begin{equation}
    \spar{M_N}^t = V_N \spar{\Lambda_N}^t \spar{V_N}^{-1}
\end{equation}
it is sufficient that $\spar{V_N}^{-1}$ is a right inverse of $V_N$. 
%
Our strategy for computing this operator is using back substitution.
%
For ease of notation, consider the inverse of $V_N$ as a collection of row vectors
\begin{equation}
    \spar{V_N}^{-1} = \begin{bmatrix}
        U_0^{(N)} \\ U_1^{(N)} \\ \vdots
    \end{bmatrix}
\end{equation} 


Furthermore, since $V_N$ is upper diagonal, we assume that $\spar{V_N}^{-1}$ is also upper-diagonal.







In other words, we aim to solve the following system
\begin{equation}
    V_N\, U_k^{(N)} = e_k
    \label{b:01}
\end{equation}
for $k = 0, 1, \dots$. This operation is simplified since 
$V_N$ is a lower diagonal matrix, and thus $\spar{V_N}^{-1}$ is too.

To be specific, the $j$-th column of equation \eqref{b:01} is
\begin{equation}
    \sum_{m=j}^k V(j,m)\, U_k^{(N)}(m) = 
    \begin{cases}
        1, &\text{if } j=k \\
        0, &\text{otherwise}
    \end{cases}
    \label{eq:b02}
\end{equation}

The particular case of the $k$-th column leads to
\begin{equation}
    V(k,k)\, U_k^{(N)}(k) = 
    1
\end{equation}
but $V(k,k) = 1$, and so $U_k^{(N)}(k) = 1$.

For $j<k$, the back substitution steps consist in computing iteratively
\begin{equation}
    U_k^{(N)}(j) = \sum_{m=j+1}^k V(j,m)\, U_k^{(N)}(m)
\end{equation}
which, after some ad hoc induction, leads to
\begin{align}
U_k^{(N)}(j)
    &=
    \ppar{\frac{N+j}{N}}^{k-j} \binom{k}{j}
    \label{eq:b03}
\end{align}

In lieu of a detailed description of the derivation method, which was mostly seeking patterns visually, we prove that the expression in \eqref{eq:b03} satisfies the equation \eqref{eq:b02}.
%
This confirmation is only necessary for $0\leq j<k$, since the case $j=k$ is trivial and $U_k^{(N)}(j)=0$ for $j> k$.
\begin{align*}
    \sum_{m=j}^k V(j,m)\, U_k^{(N)}(m)
    &=
    \sum_{m=j}^k 
    \ppar{ (-1)^{m-j} \ppar{\frac{N+m}{N}}^{m-j} \frac{N+j}{N+m} \binom{m}{j} }
    \\
    &\phantom{=}
    \phantom{\sum_{m=j}^k} \cdot
    \ppar{\ppar{\frac{N+m}{N}}^{k-m} \binom{k}{m}}
\end{align*}






\section{Expected Value of Matrix Product}
\label{ap:product}

Recall that the goal of using the eigendecomposition of the transition matrix is to compute the expected density in an efficient manner.
%
In other words, the goal is to perform the following computation
\begin{equation}
    m_N(\pi, \beta) = 
    {\nu_N^T 
    V_N}\, \spar{\Lambda_N}^{\beta N} \spar{V_N}^{-1}\,e_{\pi N}
\end{equation}

%With the eigendecomposition at hand, the computations are performed.
%
To ease the computation further, it is carried by parts, starting with the product $\nu_N^T V_T$,
\begin{align*}
    \spar{\nu_N^T V_N}(m) &=
    \sum_{k=0}^\infty \spar{\nu_N}(k) \cdot V_N(k,m) \\
    &=
    \sum_{k=0}^m
    \ppar{\frac{k}{N+k}}
    \ppar{(-1)^{m-k} \ppar{\frac{N+m}{N}}^{m-k} \cdot \frac{N+k}{N+m} \binom{m}{k}}
    \\
    &=
    \frac{1}{N+m}
    \sum_{k=0}^m
    \binom{m}{k}
    \ppar{- \frac{N+m}{N}}^{m-k} k
    \\
    &=
    \frac{m}{N+m}
    \sum_{k=1}^m
    \binom{m-1}{k-1}
    \ppar{- \frac{N+m}{N}}^{m-k}
    \\
    &=
    \frac{m}{N+m}
    \sum_{k'=0}^{m-1}
    \binom{m-1}{k'}
    \ppar{- \frac{N+m}{N}}^{m-1-k'}
    \\
    &=
    \frac{m}{N+m}
    \ppar{1-\frac{N+m}{N}}^{m-1}
    \\
    &=
    \frac{m}{N+m}
    \ppar{-\frac{m}{N}}^m \ppar{-\frac{N}{m}}
    \\
    &=
    (-1)^{m-1} \ppar{\frac{m}{N}}^m \frac{N}{N+m}
\end{align*}

The matrix $\spar{\Lambda_N}^{\beta N}$ is trivially a diagonal matrix whose entries are given by
\begin{equation}
    \spar{\ppar{\Lambda_N}^{\beta N}}(m,m)
    =
    \ppar{\frac{N}{N+m}}^{\beta N}
\end{equation}

Thus the product $\nu_N^T V_T \spar{\Lambda_N}^{\beta N}$ is almost trivial
\begin{equation}
    \spar{\nu_N^T V_T \spar{\Lambda_N}^{\beta N}}(m)
    =
    (-1)^{m-1} \ppar{\frac{m}{N}}^m
    \ppar{\frac{N}{N+m}}^{\beta N+1}
\end{equation}

The product $\spar{V_N}^{-1}\,e_{\pi N}$ is equivalent to selecting the $\pi N$-th column of $\spar{V_N}^{-1}$, which is given by
\begin{align}
\spar{V_N}^{-1}(m, \pi N)
    &=
    \begin{cases}
        \ppar{\frac{N+m}{N}}^{\pi N-m} \binom{\pi N}{m},
        &\text{if } m<\pi N \\
        0, &\text{otherwise}
    \end{cases}
\end{align}

%\begin{align*}
%    \spar{\nu_N^T V \Lambda^{(1+\beta) \alpha N} }(m) &=
%    (-1)^{m-1} \ppar{\frac{m}{N}}^m 
%    \ppar{\frac{N}{N+m}}^{(1+\beta) \alpha N+1}
%    %\par{\frac{N}{N+m}}^{(1+\beta) \alpha N}
%\end{align*}

Finally, the quantity of interest is computed
\begin{align*}
    m_N(\pi, \beta) 
    &=
    \ppar{\nu_N^T V_N \spar{\Lambda_N}^{\beta N}} \ppar{\spar{V_N}^{-1}\,e_{\pi N}}
    \\
    &=
    \sum_{m=0}^\infty 
    \spar{\nu_N^T V_T \spar{\Lambda_N}^{\beta N}}(m) \cdot
    \spar{V_N}^{-1}(m, \pi N)
    \\
    &=
    \sum_{m=0}^{\pi N}
    \ppar{(-1)^{m-1} \ppar{\frac{m}{N}}^m
    \ppar{\frac{N}{N+m}}^{\beta N+1}}
    \\
    &\phantom{=}
    \phantom{\sum_{m=0}^{\pi N}} \cdot
    \ppar{\ppar{\frac{N+m}{N}}^{\pi N-m} \binom{\pi N}{m}}
    \\
    &=
    \sum_{m=0}^{\pi N}
    (-1)^{m-1} \binom{\pi N}{m}
    \ppar{\frac{m}{N}}^m \ppar{\frac{N+m}{N}}^{-m}
    \\
    &\phantom{=}
    \phantom{\sum_{m=0}^{\pi N}} \cdot
    \ppar{\frac{N}{N+m}}^{\beta N+1}
    \ppar{\frac{N+m}{N}}^{\pi N}
    \\
    &=
    \sum_{k=0}^{\pi N}
    (-1)^{k+1} 
    \binom{\pi N}{k}
    \ppar{\frac{k}{N+k}}^k
    \ppar{\frac{N+k}{N}}^{\spar{\pi-\beta} N -1}
\end{align*}

\section{Pointwise Convergence of Coefficients}
\label{ap:coeff}

Now that we have the following expression for the quantity of interest,
\begin{equation}
    m_N(\pi, \beta) =
    \sum_{k=0}^{\pi N}
    (-1)^{k+1} 
    \binom{\pi N}{k}
    \ppar{\frac{k}{N+k}}^k
    \ppar{\frac{N+k}{N}}^{\spar{\pi-\beta} N -1}
\end{equation}
it is relevant to study its behavior as $N\rightarrow \infty$. 
%
To be precise, we are interested in the following limit
\begin{multline}
    \lim_{N \rightarrow \infty}
    (-1)^{k+1} 
    \binom{\pi N}{k}
    \ppar{\frac{k}{N+k}}^k
    \ppar{\frac{N+k}{N}}^{\spar{\pi-\beta} N -1}
    \\
    =
    (-1)^{k+1}
    \ppar{\lim_{N \rightarrow \infty}
    \binom{\pi N}{k}
    \ppar{\frac{k}{N+k}}^k}
    \ppar{\lim_{N \rightarrow \infty}
    \ppar{\frac{N+k}{N}}^{\spar{\pi-\beta} N -1}}
\end{multline}

The second limit can be dealt with easily
\begin{align*}
    \lim_{N \rightarrow \infty}
    \ppar{\frac{N+k}{N}}^{\spar{\pi-\beta} N -1}
    &=
    \lim_{N \rightarrow \infty}
    \ppar{\ppar{1+\frac{k}{N}}^N}^{\spar{\pi - \beta}}
    \ppar{\frac{N+k}{N}}
    =
    e^{\spar{\pi - \beta}k}
\end{align*}

The first limit can be approximated using a technique used on the Poisson approximation of binomial random variables. 
%
To be precise
\begin{align*}
    \lim_{N \rightarrow \infty}
    \binom{\pi N}{k}
    \ppar{\frac{k}{N+k}}^k
    &=
    \lim_{N \rightarrow \infty}
    \frac{\ppar{\pi N}!}{k!\, \ppar{\pi N-k}!} \cdot 
    \frac{k^k}{\ppar{N+k}^k}
    \\
    &=
    \frac{k^k}{k!}
    \lim_{N \rightarrow \infty}
    \frac{\ppar{\pi N}!}{\ppar{\pi N-k}!\, \ppar{N+k}^k}
    \\
    &=
    \frac{k^k}{k!}
    \lim_{N \rightarrow \infty}
    \frac{\ppar{\pi N-k} \cdots \ppar{\pi N} }{\ppar{N+k}^k}
    \\
    &=
    \frac{k^k}{k!}
    \lim_{N \rightarrow \infty}
    \ppar{\frac{\pi N-k}{N+k}} \cdots \ppar{\frac{\pi N}{N+k}}
    \\
    &=
    \frac{k^k}{k!}
    \pi ^k
\end{align*}

Thus, the intended limit is found
\begin{multline}
    \lim_{N \rightarrow \infty}
    (-1)^{k+1} 
    \binom{\pi N}{k}
    \ppar{\frac{k}{N+k}}^k
    \ppar{\frac{N+k}{N}}^{\spar{\pi-\beta} N -1}
    =
    (-1)^{k+1}
    \frac{k^k}{k!}
    \pi ^k
    e^{\spar{\pi - \beta}k}
\end{multline}

\section{Convergence of Density Function}
\label{ap:convergence}

the proposed density function is given by
\begin{equation}
    \mu(\alpha, \beta) =
    \sum_{k=0}^{\infty}
    (-1)^{k+1} \frac{k^k}{k!} 
    \ppar{\pi
    e^{\spar{\pi-\beta}} }^k
\end{equation}

The convergence of this series is investigated by using the ratio test over its coefficients.
\begin{align*}
    1 &>
    \lim_{k\rightarrow \infty}
    \frac{\aabs{
    (-1)^{\ppar{k+1}+1} \frac{\ppar{k+1}^{\ppar{k+1}}}{\ppar{k+1}!} 
    \ppar{\pi
    e^{\spar{\pi-\beta}} }^{\ppar{k+1}}
    }}{\aabs{
    (-1)^{k+1} \frac{k^k}{k!} 
    \ppar{\pi
    e^{\spar{\pi-\beta}} }^k
    }}
    \\
    &=
    \lim_{k\rightarrow \infty}
    {
    \frac{\ppar{k+1}^{\ppar{k+1}}\, k! }{{k}^{k} \ppar{k+1}!}
    {\pi e^{\spar{\pi-\beta}} }
    }
    \\
    &=
    \lim_{k\rightarrow \infty}
    \ppar{\frac{k+1}{k}}^k {\pi e^{\spar{\pi-\beta}} }
    \\
    &=
    {\pi e^{\spar{\pi-\beta}+1} }
\end{align*}

Taking logarithm, the expression is simplified to
\begin{equation}
    0 > \ln{\pi} + \spar{\pi-\beta}+1
\end{equation}
which is equivalent to
\begin{equation}
    \beta > \pi + 1 + \ln{\pi}
\end{equation}

\section{Density Function and the Lambert W Function}
\label{ap:identities}

The Lambert W Function, $W: \R_+\rightarrow \R$, is defined\footnote{This definition has a domain limited for the scope of this work.} by the following relation
\begin{equation}
    W(z) e^{W(z)} = z
\end{equation}

Using implicit differentiation over this expression leads to an identity that will be used later.
\begin{align*}
    1 &=
    \frac{d}{d z} \spar{W(z) e^{W(z)}}
    \\ &=
    W'(z)\, e^{W(z)} + W(z)\, e^{W(z)}\, W'(z)
    \\ &=
    W'(z) \ppar{1+W\ppar{z}} e^{W(z)}
    \\ &=
    W'(z) \ppar{1+W\ppar{z}} \frac{z}{W(z)}
\end{align*}

After simplification, we have the following identity
\begin{equation}
    \frac{d}{dz} W(z) 
    =
    \frac{1}{z} \cdot \frac{W(z)}{1+W(z)}
\end{equation}
Although this identity seems to come out of nowhere, it is motivated by the results obtained by Hess and Poliset \cite{hess2023}.

Now, another important identity is the Taylor expansion of the Lambert W function, given by
\begin{align}
    W(z) 
    &=
    \sum_{k=1}^\infty 
    \frac{(-k)^{k-1}}{k!} z^k
\end{align}

With both identities at hand, we can establish the following computation
\begin{align*}
    \frac{W(z)}{1+W(z)}
    &=
    z \cdot \frac{d}{dz} W(z)
    \\
    &=
    z \cdot \frac{d}{dz} \ppar{\sum_{k=1}^\infty 
    \frac{(-k)^{k-1}}{k!} z^k}
    \\
    &=
    z \cdot {\sum_{k=1}^\infty 
    \ppar{-1}^{k-1}
    \frac{k^{k-1}}{\ppar{k-1}!} z^{k-1}}
    \\
    &=
    {\sum_{k=1}^\infty 
    \ppar{-1}^{k-1}
    \frac{k^{k}}{{k}!} z^{k}}
\end{align*}

The resulting identity is
\begin{equation}
    \frac{W(z)}{1+W(z)}
    =
    {\sum_{k=1}^\infty 
    \ppar{-1}^{k-1}
    \frac{k^{k}}{{k}!} z^{k}}
\end{equation}

Notice the resemblance with the density function
\begin{equation}
    \mu(\alpha, \beta) =
    \sum_{k=0}^{\infty}
    (-1)^{k+1} \frac{k^k}{k!} 
    \ppar{\pi e^{\spar{\pi-\beta}} }^k
\end{equation}
which is why we can state the following
\begin{equation}
    \mu(\alpha, \beta) =
    \frac{W\ppar{\pi e^{\spar{\pi-\beta}}}}{1+W\ppar{\pi e^{\spar{\pi-\beta}}}}
\end{equation}



%%%%%%%%%%%%%%%%%%%%%%%%%%%%%%%%%%%%%%%%%%%%%%%%%%%%%%%%%%%%%%%%%%%%%%%%%%%%%%%
%  REFERENCES
%%%%%%%%%%%%%%%%%%%%%%%%%%%%%%%%%%%%%%%%%%%%%%%%%%%%%%%%%%%%%%%%%%%%%%%%%%%%%%%

\bibliographystyle{plain} % We choose the "plain" reference style
\bibliography{lambert_ref} % Entries are in the refs.bib file

\end{document}
